%%%%%%%%%%%%%%%%%%%%%%%%%%%%%%%%%%%%%%%%%
% Long Professional Curriculum Vitae
% LaTeX Template
% Version 1.1 (9/12/12)
%
% This template has been downloaded from:
% http://www.latextemplates.com
%
% Original author:
% Rensselaer Polytechnic Institute (http://www.rpi.edu/dept/arc/training/latex/resumes/)
%
% Important note:
% This template requires the res.cls file to be in the same directory as the
% .tex file. The res.cls file provides the resume style used for structuring the
% document.
%
%%%%%%%%%%%%%%%%%%%%%%%%%%%%%%%%%%%%%%%%%

%----------------------------------------------------------------------------------------
%	PACKAGES AND OTHER DOCUMENT CONFIGURATIONS
%----------------------------------------------------------------------------------------

\documentclass[10pt]{res} % Use the res.cls style, the font size can be changed to 11pt or 12pt here

\usepackage{helvet} % Default font is the helvetica postscript font
%\usepackage{newcent} % To change the default font to the new century schoolbook postscript font uncomment this line and comment the one above
\usepackage[colorlinks,linkcolor=blue]{hyperref}

\newsectionwidth{0pt} % Stops section indenting

\begin{document}

%----------------------------------------------------------------------------------------
%	YOUR NAME AND ADDRESS(ES) SECTION
%----------------------------------------------------------------------------------------

\name{\huge Weilun Sun \\ \\ \\} % Your name at the top

% If you don't want one of the addresses, simply remove all the text in the first or second \address{} bracket

\address{{\bf Email} \\ sunweilunjwilson@gmail.com}

\address{{\bf Homepage} \\ \url{http://sunweilun.github.com}}

%----------------------------------------------------------------------------------------

\begin{resume}

%----------------------------------------------------------------------------------------
%	EDUCATION SECTION
%----------------------------------------------------------------------------------------

\section{\centerline{EDUCATION}} 

\noindent\rule{\textwidth}{2pt}

{\bf \large Bachelor of Engineering} \hfill 2010 -- 2014 (expected) \\ 
Computer Science \& Technology \\ 
Tsinghua University, Beijing, China \\
Overall GPA: 88/100   \hfill Official Overall Ranking: \#21 out of 100 students

%----------------------------------------------------------------------------------------
 
\vspace{0.1in} % Some whitespace between sections

%----------------------------------------------------------------------------------------
%	RESEARCH EXPERIENCE SECTION
%----------------------------------------------------------------------------------------

\section{\centerline{RESEARCH EXPERIENCE}} 

\noindent\rule{\textwidth}{2pt}

{\bf \large Anisotropic Spherical Gaussians} \hfill Febuary -- May, 2013 \\
SIGGRAPH Asia 2013 Technical Paper \\
Graphics \& Geometry Computing Group, TNList, Tsinghua University \\
Mentor: \href{http://cg.cs.tsinghua.edu.cn/people/~kun/}{Dr. Kun Xu}

\begin{itemize} \itemsep -2pt % Reduce space between items
\item Investigated the form of Anisotropic Spherical Gaussian (ASG for short).
\item Implemented a {\sl Precomputed Radiance Transfer} rendering program based on theories in the paper.
\item Proofread derivation of the closed form integral and the convolution expression of ASGs.
\item Made most of result figures in the paper.
\end{itemize}

{\bf \large Sketch2Scene} \hfill September, 2012 -- January, 2013 \\
SIGGRAPH 2013 Technical Paper \\
Graphics \& Geometry Computing Group, TNList, Tsinghua University \\
Mentor: \href{http://cg.cs.tsinghua.edu.cn/people/~kun/}{Dr. Kun Xu}

\begin{itemize} \itemsep -2pt % Reduce space between items
\item Reproduced a single-model retriever based on paper
\href{http://cybertron.cg.tu-berlin.de/eitz/pdf/2012_siggraph_sbsr.pdf}{\sl Sketch-Based Shape Retrieval}.
\item Implemented part of the GUI of the project system.
\item Discussed the co-retrieval methods in the paper.
\item Provided the co-arrangement algorithm in the paper.
\end{itemize}

{\bf \large Graduation Project of Yan Gu} (second year PhD candidate at CMU now) \hfill March -- May, 2012 \\
Student Research Training Program \\
Graphics \& Geometry Computing Group, TNList, Tsinghua University \\
Mentor: \href{http://cg.cs.tsinghua.edu.cn/people/~kun/}{Dr. Kun Xu}

\begin{itemize} \itemsep -2pt % Reduce space between items
\item Reproduced main algorithms of SIGGRAPH Asia 2009 paper {\sl All-Frequency Rendering of Dynamic, Spatial-Varying Reflectance}.
\end{itemize}

%----------------------------------------------------------------------------------------

\vspace{0.1in} % Some whitespace between sections

%----------------------------------------------------------------------------------------
%	PUBLICATIONS SECTION
%----------------------------------------------------------------------------------------

\section{\centerline{PUBLICATIONS}} 

\noindent\rule{\textwidth}{2pt}

\vspace{3pt}


\begin{itemize}
\item ``Anisotropic Spherical Gaussians,"\\
Proceedings of SIGGRAPH Asia 2013, ACM Transactions on Graphics 32(6), 209:1 - 209:11, 2013.\\
\href{http://cg.cs.tsinghua.edu.cn/people/~kun/}{Kun Xu}, {\bf Wei-Lun Sun}, \href{http://flycooler.com}{Zhao Dong},
Dan-Yong Zhao, Run-Dong Wu, \href{http://cg.cs.tsinghua.edu.cn/prof_hu.htm}{Shi-Min Hu}
\item ``Sketch2Scene: Sketch-based Co-retrieval and Co-placement of 3D Models,"\\
Proceedings of SIGGRAPH 2013, ACM Transactions on Graphics 32(4) , 123:1--123:12, 2013.\\
\href{http://cg.cs.tsinghua.edu.cn/people/~kun/}{Kun Xu}, Kang Chen,
\href{http://sweb.cityu.edu.hk/hongbofu/index.htm}{Hong-Bo Fu}, {\bf Wei-Lun Sun},
\href{http://cg.cs.tsinghua.edu.cn/prof_hu.htm}{Shi-Min Hu}
\end{itemize}

%----------------------------------------------------------------------------------------

\vspace{0.1in} % Some whitespace between sections

%----------------------------------------------------------------------------------------
%	PRESENTATIONS SECTION
%----------------------------------------------------------------------------------------

\section{\centerline{PRESENTATIONS}} 

\noindent\rule{\textwidth}{2pt}

SIGGRAPH Asia 2013 Technical Paper for {\sl Anisotropic Spherical Gaussians} \hfill November 22\textsuperscript{nd}, 2013
SIGGRAPH Asia 2013 Fastforward for {\sl Anisotropic Spherical Gaussians} \hfill November 19\textsuperscript{th}, 2013

%----------------------------------------------------------------------------------------

\vspace{0.1in} % Some whitespace between sections

%----------------------------------------------------------------------------------------
%	SMALL PROJECT SECTION
%----------------------------------------------------------------------------------------

\section{\centerline{SMALL PROJECTS}} 

\noindent\rule{\textwidth}{2pt}

{\bf \large Experiment of Clustering Methods} \hfill May, 2013 \\
Course Project of {\sl Introduction to Machine Learning}

\begin{itemize} \itemsep -2pt % Reduce space between items
\item Replaced k-means clustering used in paper \href{http://cybertron.cg.tu-berlin.de/eitz/pdf/2012_siggraph_sbsr.pdf}{\sl Sketch-Based Shape Retrieval}
with different clustering methods including k-medoids and fitting Spherical Gaussians with EM algorithm.
\item Derived approximate formulas needed to fit Spherical Gaussians with EM algorithm.
\item Made simple comparisons among different methods by statistics and retrieval results.
\item Rearranged code written for \href{http://sweb.cityu.edu.hk/hongbofu/projects/sketch2scene_sig13/}{\sl Sketch2Scene} and implemented a complete software with GUI.
\end{itemize}

{\bf \large Basketball Shooting Game} \hfill January 11\textsuperscript{th} -- January 13\textsuperscript{th}, 2013 \\
Course Project of {\sl Computer Graphics Real Time and Animation}
\begin{itemize} \itemsep -2pt % Reduce space between items
\item Implemented rigid body collision simulation between a sphere and fixed objects in any shape with friction under gravity field.
\item Simulated hoop net by rigid body spheres connected by weightless springs.
\item Created a complete basketball shooting game.(Cooperated with my classmate Yi-Ning Liu)
\end{itemize}

{\bf \large Fantastic Drummer} \hfill October -- December, 2012 \\
Course Project of {\sl Principles of Signal Processing}
\begin{itemize} \itemsep -2pt % Reduce space between items
\item Leader of our group.
\item Implemented drum sound extraction and classification algorithm by Matlab.\\ (Cooperated with my classmate Iat-Chong Chan)
\item Implemented a game like Taiko no Tatsujin on ios, but can turn any input song with percussion instruments into a playable game level.(Cooperated with my classmate Yi-Ning Liu)
\item Came up with the idea.
\end{itemize}

%----------------------------------------------------------------------------------------

\vspace{0.1in} % Some whitespace between sections

%----------------------------------------------------------------------------------------
%	COMPUTER SKILLS SECTION
%----------------------------------------------------------------------------------------

\section{\centerline{COMPUTER SKILLS}}

\noindent\rule{\textwidth}{2pt}

{\bf Programming Languages:} c/c++, Java, Python, Matlab\\
{\bf Softwares \& Applications:} OPENCV, OPENGL, GLSL, QT, FLTK, Android, CUDA\\
{\bf Operating Systems:} Windows, Linux

%----------------------------------------------------------------------------------------

\vspace{0.1in} % Some whitespace between sections

%----------------------------------------------------------------------------------------
%	HONORS AND AWARDS SECTION
%----------------------------------------------------------------------------------------

\section{\centerline{HONORS AND AWARDS}} 

\noindent\rule{\textwidth}{2pt}

\vspace{-5pt} % Reduce space between section title and contents

\begin{center}
2nd Place of Tsinghua Talent Show \hfill 2012\\
\begin{itemize}
\item Performed street soccer on stage.
\end{itemize}
First Prize of Beijing Physics Olympiad for Undergraduate Students \hfill 2011
\end{center}

%----------------------------------------------------------------------------------------

\end{resume} 
\end{document}